
C-FFS is ExOS's default disk-resident file system.  In addition to the
fact that it is implemented as a library, ExOS's C-FFS differs from standard
UNIX file systems in that it employs cute techniques (described in
\cite{Ganger97}) for improving small file performance.  C-FFS has gone
thru a number of iterations, and it has departed from the version described
in the USENIX paper in order to improve performance and reduce complexity.
In particular, embedded inodes are never moved (allowing them to have
inode numbers that never change and directly specify their on-disk lcoations),
which makes several aspects simpler. 

In keeping with the spirit of a library file system, C-FFS tries to minimize
its use of shared state amongst processes.  By taking advantage of the
bc registry, it can piggyback file system-level sharing on the sharing of
underlying disk blocks.  For example, there is no ``in-core'' inode table
as exists in most UNIX systems.  Instead, processes map the bc pages that
cache copies of the disk inodes, maintain private tables for finding the
right disk inode structures in memory, and work with the disk inodes
directly (either via protected methods or via direct write access).
Similarly, each process uses its own private logical index
(i.e., $<$dev,fileno,offset$>$ to memory location) and relies on the
bc registry for a shared physical index (i.e., $<$dev,blkno$>$ to
memory page).  Finally, C-FFS makes heavy use of ExOS's
software (application-level) critical section mechanism to avoid the use
of locks.  (This last point remains an issue of debate.)

The C-FFS code base is quite a mess at this point, because it is simultaneously
supporting a number of different design options.  There are flags for
(1) using XN or using the raw disk,
(2) using the kernel-resident file system protected methods
(see sys/xok/fsprot.c) or using direct write access to update metadata
structures,
(3) using explicit grouping or not,
(4) using embedded inodes or not, and
(5) pretending to use soft updates (i.e., simply using delayed writes for
everything) or not.
In addition, insertion of support for XN (which, itself, is only partially
in place) was rushed and is therefore messy.
Finally, we have intended to reorganize C-FFS to be more asynchronous and
flexible, and the code base reflects a partial transition of this type.
Sorry.

All of that said, ExOS's C-FFS supports most of the POSIX file system
operations and provides a fairly solid base on which to work.  Most of
the crucial administrative utilities (e.g., newfs, mount, syncer, fsck)
work as one would expect.  The three main short-comings are that
(1) there is no support for partitions,
(2) only one C-FFS file system can be mounted at a time, and
(3) XN does not support fsck (and, therefore, C-FFS over XN also does not
support fsck).  ExOS's C-FFS over a raw disk does support fsck.

