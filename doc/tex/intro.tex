\section{Introduction}

This guides attempts to present a high-level view of the XOK/ExOS
source code. The idea is that this guide should make understanding the
existing source code easier. However, this guide does not attempt to
replace the source code as the prime source of information as to how the
system works.

Each of the following sections describes one sub-system such as memory
management or disk-multiplexing. Sections about a hardware resource,
such as memory management, contain a section on how XOK abstracts this
resource and then how ExOS manages it. Other sections are purely about
ExOS's emulation of some Unix feature such as processes. This document
is still a very rough draft and so significant pieces of what we
would like to ultimately be here are not here yet.

\subsection {Implementation Notes}

This section contains misc notes about the system as a whole. First, 
don't even try to use anything other than GCC. While almost all
of the code is in C, there is still a good amount of inline
asm sprinkled throughout as well as some other GCC-isms such
as \_\_attribute\_\_ etc.

Second, many of the kernel data structures have been mapped read-only
into each process. This avoids having to decide what information
to expose to applications and which to hide. This is done by allocating
such kernel data structures in a set of pages that are mapped by a single
page directory entry (PDE).  This PDE can then be inserted into
each application's page directory and marked read-only. 

Third, each system call usually takes an explicit argument of
which environment to operate on and a capability(s) proving
access to the resources being manipulated. To save programmer
effort and speed system calls, each system call usually has
an abbreviated form that defaults to operating on the caller's
environment. Thus, there are typically two variants of each
system call: {\tt sys\_foo} and {\tt sys\_self\_foo}.

\subsection {Source Tree Organization}

The source tree for the exokernel has grown large and somewhat disorganized
over time.  It includes the xok kernel, the libraries (ExOS and others),
and most ported applications.

The main subtrees are:

\begin{itemize}

\item {\bf bin/} : most of the applications that have been ported to the
exokernel appear here.  Each gets its own subdirectory (which is listed
in bin/GNUmakefile).  Adding a new application is fairly straightforward,
but does require understanding the basic aspects of the makefiles used
in the exokernel source tree.  The best approach is to look at an example
or two (good ones are: diff and csh).  Some high bits include:
(1) bin/GNUmakefile should be kept in alphabetical order, except that
"shlib" must remain first since the others depend on it,
(2) only one real target (e.g., an application program) can be made
in a directory, so extra subdirectories should be added to bin/newapp/
to support extras.

\item {\bf test/} : some other applications, usually short-term test programs,
are located here.  The structure matches that of bin/.

\item {\bf benchmarks/} : some benchmark programs that have been used for
various papers and regression tests are located here.  The structure
macthes that of bin/.

\item {\bf doc/} : documentation on a number of specific topics related to
working with the exokernel source tree and system.  Under doc/tex/, there
are latex sources for two useful get-started documents: this one
(internals.tex) on the internal/source structure and one (guide.tex)
that discusses configuring, making and booting.

\item {\bf ebin/} : applications that can be linked into the kernel as
the first program run at boot time appear here.  The guide.tex
documentation discusses the options further.

\item {\bf include/} : most of the system-wide include files appear here.
Most of them are taken from (and match) the corresponding OpenBSD files.
The ExOS-specific header files are in include/exos/, and the xok-specific
header files are in sys/xok/.

\item {\bf lib/} : the libraries (ExOS and otherwise) appear here.  Most
of them are taken directly from OpenBSD.  The main exceptions are: alfs,
xio, and libexos.  alfs is an old version of an application-level file
system, which is only used by the Cheetah web server at this point.  The
main file system, C-FFS, appears at lib/libexos/fd/cffs/.  xio is where
the TCP/IP code (up to TCP sockets) lives.  It is organized in such a way
that specializing it to application-specific purposes is strightforward.
libexos is where the bulk of the ExOS code lives.  libwafer is an example
of the minimal things a library OS must do.

\item {\bf sys/} : the xok kernel lives here.  It is broken into the following
main subdirectories: conf/ (autoconfiguration of syscall, interrupt and
exception dispatch tables), dev/ (device drivers), dpf/ (the dynamic packet
filter; see networking section), kern/ (core kernel stuff), machine/
(xok-specific, xok-internal, machine-specific header files),
ubb/ (the XN code; see disk section), vcode/ (dynamic code generation
support needed for dpf, xn and wk predicates), xok/ (header files that are
both system-wide and xok-specific), xok\_include/ (header files that are
xok-specific and internal), xoklibc/ (libc-style functions, like string
manipulation, for inside the xok kernel).

\item {\bf tools/} : a variety of tools related to the exokernel system.

\item {\bf tree/} : base files and directory structure that gets installed
into the NFS-mounted root partition during a "gmake install".  Prime
examples include /etc/* files and /home directores (and their contents).

\end{itemize}

