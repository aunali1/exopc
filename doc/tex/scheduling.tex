\section{CPU and Scheduling}

The kernel multiplexes the processor by dividing it into time slices of
a fixed number of ticks. An environment may then allocate time slices
in order to be scheduled on the processor. By default, the kernel runs
each time slice for the specified number of ticks and then switches to
the next time-slice, in order. A time slice may also be marked as
sleeping in which case the kernel will skip it and not deduct any
ticks from that slice. When the kernel has cycled through all slices
in the system it replenishes the ticks of each slice up to a maximum.

A process may donate the rest of its time slice to either a specified
environment or to any environment via the {\tt yield} ExOS
call\footnote{see include/exos/process.h}. 

A process may also put itself to sleep by setting it's {\tt u\_status}
to {\tt U\_SLEEP} and {\em another} process can wake that process up
by setting the process's {u\_status} to {\tt U\_RUN}. Note the problem
that once a process goes to sleep it has to rely on some other process
to wake it up. To get around this problem, wakeup predicates can be
used.

\subsection {Wakeup Predicates}

Wakeup predicates are predicates on the system state that the kernel
periodically evaluates. When the predicates are true the environment
associated with them has its {\tt u\_status} set to {\tt U\_RUN}. Rather
than use the raw kernel interface for predicates you should use ExOS's
wrappers.\footnote{see lib/libexos/os/uwk.c, sys/kern/wk.c, and sys/xok/wk.h}

Wakeup predicates are specified using a simple language with only an
unsigned integer type. The predicate is a sum-of-terms in which
each term is a product of ``value op value'' triples in
postfix. Each value can either be a literal or the contents of a
memory location. The op can be equal, not equal, less then, etc.

Additionally, tags can be associated with each product so that when a
predicate evaluates to true and a process is moved from sleeping to
runnable the process can tell which product triggered the wakeup. A
tag is simply an integer specified with the {\tt WK\_TAG} instruction.
